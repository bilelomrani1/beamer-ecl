\providecommand{\main}{.}

\documentclass{beamer}
\usetheme{ecl}
\usepackage[T1]{fontenc}
\usepackage[utf8]{inputenc}
\usepackage[french]{babel}
\usepackage{ragged2e}
\usepackage{booktabs}
\usepackage{tabularx}
\usepackage{tikz}
\usetikzlibrary{calc, shapes, backgrounds, positioning}
\usetikzlibrary{arrows.meta}
\usetikzlibrary{decorations.pathmorphing}
\usetikzlibrary{snakes}
\usetikzlibrary{overlay-beamer-styles}
\usepackage{amsmath, amssymb}
\usepackage{url}
\usepackage{listings}
\usepackage{pifont}
\newcommand{\cmark}{\ding{51}}%
\newcommand{\xmark}{\ding{55}}%
\usepackage{pgfplots}
\usepgfplotslibrary{fillbetween}
\usepackage{appendixnumberbeamer}

\frenchspacing

\uselanguage{french}
\languagepath{french}
\deftranslation[to=french]{Definition}{Définition}
\deftranslation[to=french]{definition}{définition}
\deftranslation[to=french]{Theorem}{Théorème}
\deftranslation[to=french]{theorem}{théorème}

\newcommand{\backupbegin}{
   \newcounter{framenumberappendix}
   \setcounter{framenumberappendix}{\value{framenumber}}
}
\newcommand{\backupend}{
   \addtocounter{framenumberappendix}{-\value{framenumber}}
   \addtocounter{framenumber}{\value{framenumberappendix}} 
}

\title{Titre}
\subtitle{Sous-titre}
\author{Auteur 1 \and Auteur 2}
% \institute{}
\supervisor{
\begin{columns}[onlytextwidth]
  \column{.5\textwidth}
  \textbf{Président du jury} \\
  Prénom Nom \\ \vskip0.5em
  \textbf{Tuteur \& commanditaire} \\
  Prénom Nom \\ 
  \column{.5\textwidth}
  \hfill \textbf{Conseiller 1}  \\
  \hfill Prénom Nom \\ \vskip0.5em
  \hfill \textbf{Conseiller 2}  \\
  \hfill Prénom Nom 
\end{columns}
}

\begin{document}

\begin{frame}[plain]
\maketitle{}
\end{frame}

% --------- Sommaire au début de chaque partie ---------

\AtBeginSection[]{
\begin{frame}<beamer>
    \frametitle{Sommaire de la section \thesection}
    \tableofcontents[currentsection]
\end{frame}}


\section{Slides claires}
\subsection{Texte simple}
\begin{frame}{Jabberwocky}
\framesubtitle{Lewis Carroll}%
'Twas brillig, and the slithy toves\\
Did gyre and gimble in the wabe;\\
All mimsy were the borogoves,\\
And the mome raths outgrabe.\\\bigskip

“Beware the Jabberwock, my son!\\
The jaws that bite, the claws that catch!\\
Beware the Jubjub bird, and shun\\
The frumious Bandersnatch!”\\
\end{frame}

\begin{frame}[label=lists]{Listes}
\framesubtitle{Lorem ipsum dolor sit amet}
\begin{columns}[onlytextwidth]
\column{.5\textwidth}
    \begin{itemize}
    \item Nulla nec lacinia odio. Curabitur urna tellus.
    \begin{itemize}
        \item Fusce id sodales dolor. Sed id metus dui.
        \begin{itemize}
        \item Cupio virtus licet mi vel feugiat.
        \end{itemize}
    \end{itemize}
    \end{itemize}
\column{.5\textwidth}
    \begin{enumerate}
    \item Donec porta, risus porttitor egestas scelerisque video.
    \begin{enumerate}
        \item Nunc non ante fringilla, manus potentis cario.
        \begin{enumerate}
        \item Pellentesque servus morbi tristique.
        \end{enumerate}
    \end{enumerate}
    \end{enumerate}
\end{columns}
\end{frame}

\subsection{Éléments de structure}
\begin{frame}[label=simmonshall]{Blocs de texte}
    \framesubtitle{Style simple, exemples ou \alert{alertes}}
    \alert{Ce texte} est important.

    \begin{block}{Un bloc simple}
    Voici un bloc simple contenant \alert{du texte important}.
    \end{block}
    \begin{exampleblock}{Un bloc d'exemple}
    Voici un bloc d'exemple contenant \alert{du texte important}.
    \end{exampleblock}
    \begin{alertblock}{Un bloc d'alerte}
    Voici un bloc d'alerte contenant \alert{du texte important}.
    \end{alertblock}
\end{frame}

\begin{frame}[label=proof]{Définitions, théorèmes et preuve}
\framesubtitle{Tout entier divise zéro}
\begin{definition}
$\forall a,b\in\mathbb{Z}: a\mid b\iff\exists c\in\mathbb{Z}:a\cdot c=b$
\end{definition}
\begin{theorem}
$\forall a\in\mathbb{Z}: a\mid 0$
\end{theorem}
\begin{proof}[Preuve\nopunct]
$\forall a\in\mathbb{Z}: a\cdot 0=0$
\end{proof}
\end{frame}


\subsection{Mathématiques}
\begin{frame}[label=math]{Nombres et mathématiques}
\framesubtitle{Formules, équations et expressions}
\begin{columns}[onlytextwidth]
\column{.20\textwidth}
    1234567890
\column{.40\textwidth}
    $\hat{x}$, $\check{x}$, $\tilde{a}$,
    $\bar{a}$, $\dot{y}$, $\ddot{y}$
\column{.40\textwidth}
    $$ \iiint f(x,y,z)\,\mathsf{d}x\mathsf{d}y\mathsf{d}z$$
\end{columns}
\begin{columns}[onlytextwidth]
\column{.5\textwidth}
    $$\frac{1}{\displaystyle 1+
    \frac{1}{\displaystyle 2+
    \frac{1}{\displaystyle 3+x}}} +
    \frac{1}{1+\frac{1}{2+\frac{1}{3+x}}}$$
\column{.5\textwidth}
    $$F:\left| \begin{array}{ccc}
    F''_{xx} & F''_{xy} &  F'_x \\
    F''_{yx} & F''_{yy} &  F'_y \\
    F'_x     & F'_y     & 0
    \end{array}\right| = 0$$
\end{columns}
\begin{columns}[onlytextwidth]
\column{.3\textwidth}
    $$\mathop{\int \!\!\! \int}_{\mathbf{x} \in \mathbb{R}^2}
    \! \langle \mathbf{x},\mathbf{y}\rangle\,\mathsf{d}\mathbf{x}$$
\column{.33\textwidth}
    $$\overline{\overline{a\alpha}^2+\underline{b\beta}
    +\overline{\overline{d\delta}}}$$
\column{.37\textwidth}
    $\left] 0,1\right[ + \lceil x \rfloor - \langle x,y\rangle$
\end{columns}
\begin{columns}[onlytextwidth]
\column{.4\textwidth}
    \begin{eqnarray*}
    e^x &\approx& 1+x+x^2/2! + \\
        && {}+x^3/3! + x^4/4!
    \end{eqnarray*}
\column{.6\textwidth}
    $${n+1\choose k} = {n\choose k} + {n \choose k-1}$$
\end{columns}
\end{frame}

\subsection{Figures et Code Listings}
\begin{frame}[label=figs1]{Figures}
\framesubtitle{Tableaux, graphiques, et images}
\begin{table}[!b]
{\carlitoTLF % Use monospaced lining figures
\begin{tabularx}{\textwidth}{Xrrr}
    \textbf{Faculty} & \textbf{With \TeX} & \textbf{Total} &
    \textbf{\%} \\
    \toprule
    Faculty of Informatics       & 1\,716  & 2\,904  &
    59.09 \\% 1433
    Faculty of Science           & 786     & 5\,275  &
    14.90 \\% 1431
    Faculty of $\genfrac{}{}{0pt}{}{\textsf{Economics and}}{%
    \textsf{Administration}}$    & 64      & 4\,591  &
    1.39  \\% 1456
    Faculty of Arts              & 69      & 10\,000 &
    0.69  \\% 1421
    Faculty of Medicine          & 8       & 2\,014  &
    0.40  \\% 1411
    Faculty of Law               & 15      & 4\,824  &
    0.31  \\% 1422
    Faculty of Education         & 19      & 8\,219  &
    0.23  \\% 1441
    Faculty of Social Studies    & 12      & 5\,599  &
    0.21  \\% 1423
    Faculty of Sports Studies    & 3       & 2\,062  &
    0.15  \\% 1451
    \bottomrule
\end{tabularx}}
\caption{The distribution of theses written using \TeX\ during 2010--15 at MU}
\end{table}
\end{frame}
\begin{frame}[label=figs2]{Figures}
\framesubtitle{Tableaux, graphiques, et images}
\begin{figure}[b]
\centering
% Flipping a coin
% Author: cis
\tikzset{
    head/.style = {fill = none, label = center:\textsf{H}},
    tail/.style = {fill = none, label = center:\textsf{T}}}
\scalebox{0.65}{\begin{tikzpicture}[
    scale = 1.5, transform shape, thick,
    every node/.style = {draw, circle, minimum size = 10mm},
    grow = down,  % alignment of characters
    level 1/.style = {sibling distance=3cm},
    level 2/.style = {sibling distance=4cm},
    level 3/.style = {sibling distance=2cm},
    level distance = 1.25cm
    ]
    \node[shape = rectangle,
    minimum width = 6cm, font = \sffamily] {Coin flipping}
    child { node[shape = circle split, draw, line width = 1pt,
            minimum size = 10mm, inner sep = 0mm, rotate = 30] (Start)
            { \rotatebox{-30}{H} \nodepart{lower} \rotatebox{-30}{T}}
    child {   node [head] (A) {}
        child { node [head] (B) {}}
        child { node [tail] (C) {}}
    }
    child {   node [tail] (D) {}
        child { node [head] (E) {}}
        child { node [tail] (F) {}}
    }
    };

    % Filling the root (Start)
    \begin{scope}[on background layer, rotate=30]
    \fill[head] (Start.base) ([xshift = 0mm]Start.east) arc (0:180:5mm)
        -- cycle;
    \fill[tail] (Start.base) ([xshift = 0pt]Start.west) arc (180:360:5mm)
        -- cycle;
    \end{scope}

    % Labels
    \begin{scope}[nodes = {draw = none}]
    \path (Start) -- (A) node [near start, left]  {$0.5$};
    \path (A)     -- (B) node [near start, left]  {$0.5$};
    \path (A)     -- (C) node [near start, right] {$0.5$};
    \path (Start) -- (D) node [near start, right] {$0.5$};
    \path (D)     -- (E) node [near start, left]  {$0.5$};
    \path (D)     -- (F) node [near start, right] {$0.5$};
    \begin{scope}[nodes = {below = 11pt}]
        \node [name = X] at (B) {$0.25$};
        \node            at (C) {$0.25$};
        \node [name = Y] at (E) {$0.25$};
        \node            at (F) {$0.25$};
    \end{scope}
    \end{scope}
\end{tikzpicture}}
\caption{Tree of probabilities -- Flipping a coin\footnote[frame]{%
    A derivative of a diagram from \url{texample.net} by cis, CC BY 2.5 licensed}}
\end{figure}
\end{frame}


% \defverbatim[colored]\sleepSort{
% \begin{lstlisting}[language=C,tabsize=2]
% #include <stdio.h>
% #include <unistd.h>
% #include <sys/types.h>
% #include <sys/wait.h>

% // This is a comment
% int main(int argc, char **argv)
% {
%     while (--c > 1 && !fork());
%     sleep(c = atoi(v[c]));
%     printf("%d\n", c);
%     wait(0);
%     return 0;
% }
% \end{lstlisting}}
% \begin{frame}{Code listings}{An example source code in C}
% \sleepSort
% \end{frame}

\subsection{Citations and Bibliography}
\begin{frame}[label=citations]{Citations}
\framesubtitle{\TeX, \LaTeX, and Beamer}

\justifying\TeX\ is a programming language for the typesetting
of documents. It was created by Donald Erwin Knuth in the late
1970s and it is documented in \emph{The \TeX
book}~\cite{knuth84}.

In the early 1980s, Leslie Lamport created the initial version
of \LaTeX, a high-level language on top of \TeX, which is
documented in \emph{\LaTeX : A Document Preparation
System}~\cite{lamport94}. There exists a healthy ecosystem of
packages that extend the base functionality of \LaTeX;
\emph{The \LaTeX\ Companion}~\cite{MG94} acts as a guide
through the ecosystem.

In 2003, Till Tantau created the initial version of Beamer, a
\LaTeX\ package for the creation of presentations. Beamer is
documented in the \emph{User's Guide to the Beamer
Class}~\cite{tantau04}.
\end{frame}

\begin{frame}[label=bibliography]{Bibliography}
\framesubtitle{\TeX, \LaTeX, and Beamer}
\begin{thebibliography}{9}
\bibitem{knuth84}
    Donald~E.~Knuth.
    \emph{The \TeX book}.
    Addison-Wesley, 1984.
\bibitem{lamport94}
    Leslie~Lamport.
    \emph{\LaTeX : A Document Preparation System}.
    Addison-Wesley, 1986.
\bibitem{MG94}
    M.~Goossens, F.~Mittelbach, and A.~Samarin.
    \emph{The \LaTeX\ Companion}.
    Addison-Wesley, 1994.
\bibitem{tantau04}
    Till~Tantau.
    \emph{User's Guide to the Beamer Class Version 3.01}.
    Available at \url{http://latex-beamer.sourceforge.net}.
\bibitem{MS05}
    A.~Mertz and W.~Slough.
    Edited by B.~Beeton and K.~Berry.
    \emph{Beamer by example} In TUGboat,
        Vol. 26, No. 1., pp. 68-73.
\end{thebibliography}
\end{frame}

\section{Slides claires}
\subsection{Texte simple}
\begin{frame}{Jabberwocky}
\framesubtitle{Lewis Carroll}%
'Twas brillig, and the slithy toves\\
Did gyre and gimble in the wabe;\\
All mimsy were the borogoves,\\
And the mome raths outgrabe.\\\bigskip

“Beware the Jabberwock, my son!\\
The jaws that bite, the claws that catch!\\
Beware the Jubjub bird, and shun\\
The frumious Bandersnatch!”\\
\end{frame}

\begin{frame}[label=lists]{Listes}
\framesubtitle{Lorem ipsum dolor sit amet}
\begin{columns}[onlytextwidth]
\column{.5\textwidth}
    \begin{itemize}
    \item Nulla nec lacinia odio. Curabitur urna tellus.
    \begin{itemize}
        \item Fusce id sodales dolor. Sed id metus dui.
        \begin{itemize}
        \item Cupio virtus licet mi vel feugiat.
        \end{itemize}
    \end{itemize}
    \end{itemize}
\column{.5\textwidth}
    \begin{enumerate}
    \item Donec porta, risus porttitor egestas scelerisque video.
    \begin{enumerate}
        \item Nunc non ante fringilla, manus potentis cario.
        \begin{enumerate}
        \item Pellentesque servus morbi tristique.
        \end{enumerate}
    \end{enumerate}
    \end{enumerate}
\end{columns}
\end{frame}

\subsection{Éléments de structure}
\begin{frame}[label=simmonshall]{Blocs de texte}
    \framesubtitle{Style simple, exemples ou \alert{alertes}}
    \alert{Ce texte} est important.

    \begin{block}{Un bloc simple}
    Voici un bloc simple contenant \alert{du texte important}.
    \end{block}
    \begin{exampleblock}{Un bloc d'exemple}
    Voici un bloc d'exemple contenant \alert{du texte important}.
    \end{exampleblock}
    \begin{alertblock}{Un bloc d'alerte}
    Voici un bloc d'alerte contenant \alert{du texte important}.
    \end{alertblock}
\end{frame}

\begin{frame}[label=proof]{Définitions, théorèmes et preuve}
\framesubtitle{Tout entier divise zéro}
\begin{definition}
$\forall a,b\in\mathbb{Z}: a\mid b\iff\exists c\in\mathbb{Z}:a\cdot c=b$
\end{definition}
\begin{theorem}
$\forall a\in\mathbb{Z}: a\mid 0$
\end{theorem}
\begin{proof}[Preuve\nopunct]
$\forall a\in\mathbb{Z}: a\cdot 0=0$
\end{proof}
\end{frame}


\subsection{Mathématiques}
\begin{frame}[label=math]{Nombres et mathématiques}
\framesubtitle{Formules, équations et expressions}
\begin{columns}[onlytextwidth]
\column{.20\textwidth}
    1234567890
\column{.40\textwidth}
    $\hat{x}$, $\check{x}$, $\tilde{a}$,
    $\bar{a}$, $\dot{y}$, $\ddot{y}$
\column{.40\textwidth}
    $$ \iiint f(x,y,z)\,\mathsf{d}x\mathsf{d}y\mathsf{d}z$$
\end{columns}
\begin{columns}[onlytextwidth]
\column{.5\textwidth}
    $$\frac{1}{\displaystyle 1+
    \frac{1}{\displaystyle 2+
    \frac{1}{\displaystyle 3+x}}} +
    \frac{1}{1+\frac{1}{2+\frac{1}{3+x}}}$$
\column{.5\textwidth}
    $$F:\left| \begin{array}{ccc}
    F''_{xx} & F''_{xy} &  F'_x \\
    F''_{yx} & F''_{yy} &  F'_y \\
    F'_x     & F'_y     & 0
    \end{array}\right| = 0$$
\end{columns}
\begin{columns}[onlytextwidth]
\column{.3\textwidth}
    $$\mathop{\int \!\!\! \int}_{\mathbf{x} \in \mathbb{R}^2}
    \! \langle \mathbf{x},\mathbf{y}\rangle\,\mathsf{d}\mathbf{x}$$
\column{.33\textwidth}
    $$\overline{\overline{a\alpha}^2+\underline{b\beta}
    +\overline{\overline{d\delta}}}$$
\column{.37\textwidth}
    $\left] 0,1\right[ + \lceil x \rfloor - \langle x,y\rangle$
\end{columns}
\begin{columns}[onlytextwidth]
\column{.4\textwidth}
    \begin{eqnarray*}
    e^x &\approx& 1+x+x^2/2! + \\
        && {}+x^3/3! + x^4/4!
    \end{eqnarray*}
\column{.6\textwidth}
    $${n+1\choose k} = {n\choose k} + {n \choose k-1}$$
\end{columns}
\end{frame}

\subsection{Figures et Code Listings}
\begin{frame}[label=figs1]{Figures}
\framesubtitle{Tableaux, graphiques, et images}
\begin{table}[!b]
{\carlitoTLF % Use monospaced lining figures
\begin{tabularx}{\textwidth}{Xrrr}
    \textbf{Faculty} & \textbf{With \TeX} & \textbf{Total} &
    \textbf{\%} \\
    \toprule
    Faculty of Informatics       & 1\,716  & 2\,904  &
    59.09 \\% 1433
    Faculty of Science           & 786     & 5\,275  &
    14.90 \\% 1431
    Faculty of $\genfrac{}{}{0pt}{}{\textsf{Economics and}}{%
    \textsf{Administration}}$    & 64      & 4\,591  &
    1.39  \\% 1456
    Faculty of Arts              & 69      & 10\,000 &
    0.69  \\% 1421
    Faculty of Medicine          & 8       & 2\,014  &
    0.40  \\% 1411
    Faculty of Law               & 15      & 4\,824  &
    0.31  \\% 1422
    Faculty of Education         & 19      & 8\,219  &
    0.23  \\% 1441
    Faculty of Social Studies    & 12      & 5\,599  &
    0.21  \\% 1423
    Faculty of Sports Studies    & 3       & 2\,062  &
    0.15  \\% 1451
    \bottomrule
\end{tabularx}}
\caption{The distribution of theses written using \TeX\ during 2010--15 at MU}
\end{table}
\end{frame}
\begin{frame}[label=figs2]{Figures}
\framesubtitle{Tableaux, graphiques, et images}
\begin{figure}[b]
\centering
% Flipping a coin
% Author: cis
\tikzset{
    head/.style = {fill = none, label = center:\textsf{H}},
    tail/.style = {fill = none, label = center:\textsf{T}}}
\scalebox{0.65}{\begin{tikzpicture}[
    scale = 1.5, transform shape, thick,
    every node/.style = {draw, circle, minimum size = 10mm},
    grow = down,  % alignment of characters
    level 1/.style = {sibling distance=3cm},
    level 2/.style = {sibling distance=4cm},
    level 3/.style = {sibling distance=2cm},
    level distance = 1.25cm
    ]
    \node[shape = rectangle,
    minimum width = 6cm, font = \sffamily] {Coin flipping}
    child { node[shape = circle split, draw, line width = 1pt,
            minimum size = 10mm, inner sep = 0mm, rotate = 30] (Start)
            { \rotatebox{-30}{H} \nodepart{lower} \rotatebox{-30}{T}}
    child {   node [head] (A) {}
        child { node [head] (B) {}}
        child { node [tail] (C) {}}
    }
    child {   node [tail] (D) {}
        child { node [head] (E) {}}
        child { node [tail] (F) {}}
    }
    };

    % Filling the root (Start)
    \begin{scope}[on background layer, rotate=30]
    \fill[head] (Start.base) ([xshift = 0mm]Start.east) arc (0:180:5mm)
        -- cycle;
    \fill[tail] (Start.base) ([xshift = 0pt]Start.west) arc (180:360:5mm)
        -- cycle;
    \end{scope}

    % Labels
    \begin{scope}[nodes = {draw = none}]
    \path (Start) -- (A) node [near start, left]  {$0.5$};
    \path (A)     -- (B) node [near start, left]  {$0.5$};
    \path (A)     -- (C) node [near start, right] {$0.5$};
    \path (Start) -- (D) node [near start, right] {$0.5$};
    \path (D)     -- (E) node [near start, left]  {$0.5$};
    \path (D)     -- (F) node [near start, right] {$0.5$};
    \begin{scope}[nodes = {below = 11pt}]
        \node [name = X] at (B) {$0.25$};
        \node            at (C) {$0.25$};
        \node [name = Y] at (E) {$0.25$};
        \node            at (F) {$0.25$};
    \end{scope}
    \end{scope}
\end{tikzpicture}}
\caption{Tree of probabilities -- Flipping a coin\footnote[frame]{%
    A derivative of a diagram from \url{texample.net} by cis, CC BY 2.5 licensed}}
\end{figure}
\end{frame}


% \defverbatim[colored]\sleepSort{
% \begin{lstlisting}[language=C,tabsize=2]
% #include <stdio.h>
% #include <unistd.h>
% #include <sys/types.h>
% #include <sys/wait.h>

% // This is a comment
% int main(int argc, char **argv)
% {
%     while (--c > 1 && !fork());
%     sleep(c = atoi(v[c]));
%     printf("%d\n", c);
%     wait(0);
%     return 0;
% }
% \end{lstlisting}}
% \begin{frame}{Code listings}{An example source code in C}
% \sleepSort
% \end{frame}

\subsection{Citations and Bibliography}
\begin{frame}[label=citations]{Citations}
\framesubtitle{\TeX, \LaTeX, and Beamer}

\justifying\TeX\ is a programming language for the typesetting
of documents. It was created by Donald Erwin Knuth in the late
1970s and it is documented in \emph{The \TeX
book}~\cite{knuth84}.

In the early 1980s, Leslie Lamport created the initial version
of \LaTeX, a high-level language on top of \TeX, which is
documented in \emph{\LaTeX : A Document Preparation
System}~\cite{lamport94}. There exists a healthy ecosystem of
packages that extend the base functionality of \LaTeX;
\emph{The \LaTeX\ Companion}~\cite{MG94} acts as a guide
through the ecosystem.

In 2003, Till Tantau created the initial version of Beamer, a
\LaTeX\ package for the creation of presentations. Beamer is
documented in the \emph{User's Guide to the Beamer
Class}~\cite{tantau04}.
\end{frame}

\begin{frame}[label=bibliography]{Bibliography}
\framesubtitle{\TeX, \LaTeX, and Beamer}
\begin{thebibliography}{9}
\bibitem{knuth84}
    Donald~E.~Knuth.
    \emph{The \TeX book}.
    Addison-Wesley, 1984.
\bibitem{lamport94}
    Leslie~Lamport.
    \emph{\LaTeX : A Document Preparation System}.
    Addison-Wesley, 1986.
\bibitem{MG94}
    M.~Goossens, F.~Mittelbach, and A.~Samarin.
    \emph{The \LaTeX\ Companion}.
    Addison-Wesley, 1994.
\bibitem{tantau04}
    Till~Tantau.
    \emph{User's Guide to the Beamer Class Version 3.01}.
    Available at \url{http://latex-beamer.sourceforge.net}.
\bibitem{MS05}
    A.~Mertz and W.~Slough.
    Edited by B.~Beeton and K.~Berry.
    \emph{Beamer by example} In TUGboat,
        Vol. 26, No. 1., pp. 68-73.
\end{thebibliography}
\end{frame}

\end{document}
